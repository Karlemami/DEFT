\section{Résultats}

Comme mentionné en introduction, nous avons testé trois méthodes de vectorisation : TF-IDF, doc2vec et BERT
Nous avons utilisé ces vecteurs pour entraîner et comparer 4 modèles différents : Régression logistique, SVM, Random Forest et Perceptron.

\subsection{Vecteurs TF-IDF}

Les meilleurs résultats que nous ayons obtenus sont ceux avec un vectorisation tf-idf.
On peut voir sur la figure~\ref{fig:tfidf_comparison} que nous avons obtenu au mieux : 0.46 en anglais avec un SVM, 0.43 en italien avec une régression linéaire et 0.44 en français avec un SVM.

\begin{figure}[t]
  \includegraphics[width=\columnwidth]{assets/comparaison_metriques_tfidf.png}
  \caption{Résultats obtenus avec les vecteurs TF-IDF.}
  \label{fig:tfidf_comparison}
\end{figure}

\subsection{Vecteurs doc2vec}

Comme on peut le voir sur la figure~\ref{fig:doc2vec_comparison}, avec les vecteurs doc2vec, c'est le SVM qui obtient les meilleurs résultats dans les trois langues.

\begin{figure}[t]
  \includegraphics[width=\columnwidth]{assets/comparaison_metriques_doc2vec.png}
  \caption{Résultats obtenus avec les vecteurs doc2vec.}
  \label{fig:doc2vec_comparison}
\end{figure}
\subsection{Vecteurs BERT}

\begin{figure}[t]
  \includegraphics[width=\columnwidth]{assets/comparaison_metriques_bert.png}
  \caption{Résultats obtenus avec les vecteurs BERT.}
  \label{fig:bert_comparison}
\end{figure}

La figure~\ref{fig:bert_comparison} montre qu'avec les vecteurs BERT, la régression logistique obtient les meilleurs résultats dans les trois langues.




\subsection{Comparaison des méthodes de vectorisation}

Le tableau \ref{tab:comparaison_vecteurs} montre les résultats obtenus avec les trois méthodes de vectorisation. On peut voir que la méthode TF-IDF est la plus performante dans les trois langues.

\begin{table}[h]
    \centering
    \begin{tabular}{|c|c|c|c|}
        \hline
        & TF-IDF & DOC2VEC & BERT \\
        \hline
        Français & 0.44 & 0.33 & 0.36 \\
        \hline
        Anglais & 0.47 & 0.38 & 0.36 \\
        \hline
        Italien & 0.43 & 0.33 & 0.35 \\
        \hline
    \end{tabular}
    \caption{Comparison of different models}
    \label{tab:comparaison_vecteurs}
\end{table}

Le meilleur résultat a été obtenu sur l'anglais avec un SVM et des vecteurs TF-IDF, avec un score de 0.47. La figure \ref{fig:tfidf_comparison} montre la matrice de confusion correspondante.

\begin{figure}[t]
  \includegraphics[width=\columnwidth]{"./assets/confusion_matrix_en_tfidf.png"}
  \caption{Matrice de confusion pour l'anglais avec des vecteurs TF-IDF et un SVM.}
  \label{fig:tfidf_comparison}
\end{figure}